% LaTeX file for resume 
% This file uses the resume document class (res.cls)

\documentclass{res} 


\usepackage{amsmath}
\usepackage{easylist}
\usepackage{amssymb}
\usepackage{etoolbox}
\usepackage{xcolor}
\usepackage{lastpage}
\usepackage{hyperref}
\hypersetup{urlbordercolor={white},pagebordercolor={white},pdfnewwindow=true}

%\usepackage[margin=1in,height=20in]{geometry}
%citebordercolor [rgb 0 1 0]
%filebordercolor [rgb 0 .5 .5]
%linkbordercolor [rgb 1 0 0]
%menubordercolor [rgb 1 0 0]
%urlbordercolor [rgb 0 1 1]
%runbordercolor [rgb 0 .7 .7]
%allbordercolors


\usepackage{times}
%\usepackage{helvetica} % uses helvetica postscript font (download helvetica.sty)
%\usepackage{newcent}   % uses new century schoolbook postscript font 
\newsectionwidth{0pt}  % So the text is not indented under section headings
\usepackage{fancyhdr}  % use this package to get a 2 line header
\renewcommand{\headrulewidth}{0pt} % suppress line drawn by default by fancyhdr
\setlength{\headheight}{24pt} % allow room for 2-line header
\setlength{\headsep}{24pt}  % space between header and text
\setlength{\headheight}{24pt} % allow room for 2-line header


%\pagestyle{fancy}     % set pagestyle for document
\rhead{ {\it A. Pereira}\\{\it p. \thepage\  of~\pageref{LastPage}} } % put text in header (right side)
\cfoot{}                                     % the foot is empty



\newcommand{\gtxt }[1]{{\color{blue} {#1}}} 
\newtoggle{detailedVersion}
% If we want a detailed resume with individual languages used in each project and so on, toggle this flag true.
\toggletrue{detailedVersion}
%\togglefalse{detailedVersion}

\iftoggle{detailedVersion}{
\topmargin=-0.5in % start text higher on the page
}{\topmargin=-1in}

\begin{document}
\thispagestyle{empty} % this page has no header
\name{\LARGE{ARVIND A. DE MENEZES PEREIRA}\\[8pt]}  % the \\[12pt] adds a blank line after name
\address{{}\href{tel:12132704465}{+1(213)~270-4465} or \href{tel:12133421734}{+1(213)~342-1734} \\ 
\url{http://robotics.usc.edu/~ampereir} \\
\url{http://linkedin.com/in/arvindpereira}
}

\address{{}1055 Escalon Ave \\ Sunnyvale, CA 94085\\
  \href{mailto:arvind.pereira@gmail.com}{arvind.pereira@gmail.com}}

% ---- Resume begins here ----

\begin{resume}
\vspace{3pt}
\centerline{{{{\sl PhD (Computer Science), MS (Electrical Eng.), BE (Electronics \& Comm. Eng.)}}}}%   

\iftoggle{detailedVersion}{
\section{\centerline{OBJECTIVE}}
\vspace{8pt} % provide vertical space between section title and contents
\center{To utilize my skills in researching and developing the world's most amazing products}
}{}
% Core Expertise Section
\vspace{0.2in} 
\section{\centerline{ CORE EXPERTISE }}
\vspace{8pt}
\begin{table}[ht]
\centering
\begin{tabular}{lll}
\textbullet~Artificial Intelligence &  \textbullet~Robotics \& Control Systems &  \textbullet~Machine Learning \\
\textbullet~Planning under uncertainty & \textbullet~Computer Vision & \textbullet~Signal \& Image Processing \\
\textbullet~Embedded systems & \textbullet~Computer Networking & \textbullet~Data Structures \& Algorithms
\end{tabular}
\end{table}
 
 % Work Experience Section 
\vspace{0.2in} 
\section{\centerline{WORK EXPERIENCE}}

\vspace{8pt}
{\sl \href{http://www.clover.com}{\textbf{Clover Network Inc.}}} \hfill        October 2013 - Present \\
\href{http://www.clover.com/team}{\textbf{Mountain View, California}}       \hfill   \textbf{Software Engineer}
   \begin{itemize} \itemsep -2pt % reduce space between items
   \item Developed barcode scanner for Clover Point of Sales terminals \itemsep -2pt
   \begin{itemize}
   \item[$\checkmark$]  \itemsep -8pt%
    Scanner enables rapid product barcode and QR-code scanning using low-cost camera
    \iftoggle{detailedVersion}{\begin{itemize}\item[\tiny$\blacksquare$] Combines techniques from Computer Vision, Image Processing and Algorithms
   \item[\tiny$\blacksquare$] All code written in C++, Java for Android, with prototyping in Matlab
   \item[\tiny$\blacksquare$] Used hardware parallelization, OpenCV and developed algorithms for improved image enhancement
   \end{itemize}}{}
     \end{itemize}
  \item Implemented features in various core Clover software  \itemsep -2pt
   \begin{itemize}
   \item[$\checkmark$]  \itemsep -8pt%
    Added features to Clover's server, Android platform, Android apps and web dashboard
    \iftoggle{detailedVersion}
    { \begin{itemize}
       \item[\tiny$\blacksquare$] Languages used are Java, C++, C, Javascript and SQL
      \end{itemize}
    } {}
   \end{itemize}
\end{itemize}
 
\vspace{8pt}
{\sl \href{http://www.usc.edu}{\textbf{University of Southern California, Los Angeles}}} \hfill        January 2006 - September 2013 \\
\href{http://robotics.usc.edu/resl}{\textbf{Robotic Embedded Systems Laboratory}}       \hfill   \textbf{Research Assistant}
   \begin{itemize} \itemsep -2pt % reduce space between items
   \item \href{http://robotics.usc.edu/~ampereir/wordpress/?page_id=131}{Developed  path planners for glider AUVs under High Uncertainty} \itemsep -2pt
   \begin{itemize}\item[$\checkmark$]  \itemsep -8pt%
    1st planners for glider AUVs to deal with uncertainty in ocean model predictions 
    \iftoggle{detailedVersion}
    {\begin{itemize}\item[\tiny$\blacksquare$] Combines techniques from AI, Machine Learning and Robotics
   \item[\tiny$\blacksquare$] All code written in C++ and Python (STL, Boost, NumPy/SciPy)
   \item[\tiny$\blacksquare$] Used hardware parallelization and advanced data structures to speed up computations
    \end{itemize}
   } {}
   \end{itemize}
    
   \item \href{http://robotics.usc.edu/~ampereir/wordpress/?page_id=134}{Developed a long-range RF communication network for AUVs in the Southern California region}
   \begin{itemize} \item[$\checkmark$]Improved data throughput by over 24x and reduced operation cost by over 50\%
   \iftoggle{detailedVersion}{\begin{itemize}
   \item[\tiny$\blacksquare$]  Required Computer Networking, Embedded Systems and Signal Processing
   \item[\tiny$\blacksquare$]  All code written in C++ (both ARM and PC-based Linux) 
   \end{itemize}}{}
   \end{itemize}
   
    \item Developed a novel TCP-like communication protocol which adapts dynamically to sea state
   \begin{itemize} \item[$\checkmark$]1st motion-aware adaptive communication protocol
   \iftoggle{detailedVersion}{\begin{itemize}
   \item[\tiny$\blacksquare$]  Required Computer Networking, Reinforcement Learning and Signal Processing
   \item[\tiny$\blacksquare$]  All code written in C++ (both ARM and PC-based Linux) 
   \end{itemize}}{}
   \end{itemize}
   
    \item \href{http://robotics.usc.edu/~ampereir/wordpress/?page_id=365}{Developed the Guidance, Navigation and Control for Autonomous Surface Vehicles  including stereo-vision based obstacle avoidance} and \href{http://robotics.usc.edu/~ampereir/wordpress/?page_id=373}{Station-Keeping}
 \iftoggle{detailedVersion}{\begin{itemize} \item[$\checkmark$] Took only 2 months to complete autonomy from RC-control to Mission control}{}
    \iftoggle{detailedVersion}{
   \begin{itemize}%
   \item[\tiny$\blacksquare$]  Required Control Systems, Robotics, Computer Networking, Computer Vision, Image Processing
   \iftoggle{detailedVersion}{
   \item[\tiny$\blacksquare$]  All code on robot written in C++ (Linux); OpenCV for vision. GUI written in C\# .Net (Windows); Prototyping in Matlab (Control Systems Toolbox, System Identification Toolbox) and Simulink}{}
    \end{itemize}
    %\pagebreak%
    }{}
    \iftoggle{detailedVersion}{\end{itemize}}{}
 \end{itemize}

{\sl \href{http://welcome.isr.ist.utl.pt/home/}{\textbf{Institute for Systems and Robotics, Lisbon, Portugal}}} \hfill May 2008--August 2008  \\[2pt]
\href{http://www.grex-project.eu/}{\textbf{EU-GREX Project}} \hfill   \textbf{ Lead Developer}
\begin{itemize}
 \item \href{http://robotics.usc.edu/~ampereir/wordpress/?page_id=397}{Implemented GREX Interface module for EU-GREX project}
   \begin{itemize} \item[$\checkmark$] Met all objectives on time despite large pre-existing code-base, and multi-national teams
   \item[$\checkmark$] All software was successfully tested - only team with working robot at field test in July
   \iftoggle{detailedVersion}{\begin{itemize}
   \item[\tiny$\blacksquare$]  Required Control Systems, Robotics, Ability to understand and implement complex algorithms quickly
   \item[\tiny$\blacksquare$]  Coded in Visual C++ .Net (Embedded Windows PC) and Matlab for prototyping
   \end{itemize}}{}
   \end{itemize}
   \end{itemize} 
   
   \vspace{-6pt}
 
{\sl \href{http://nio.org}{\textbf{National Institute of Oceanography, Goa, India}}}       \hfill               September 2003 - July 2005 \\[2pt]
\textbf{Marine Instrumentation \& Computer Division} \hfill \textbf{Research Fellow}
 \begin{itemize} 
 \item  \href{http://robotics.usc.edu/~ampereir/wordpress/?page_id=425}{Led software development and hardware design on India's first Autonomous Underwater Vehicle (MAYA)}
 \begin{itemize} 
 \item[$\checkmark$] Wrote over 95\% of Control, Communications, Recovery, Guidance and Navigation code
  \iftoggle{detailedVersion}{\begin{itemize}
   \item[\tiny$\blacksquare$]  Required Control Systems, Robotics, Computer Networking
   \item[\tiny$\blacksquare$]  Coded in C++ (Linux), Visual C++ (Windows) for Control GUI, C (micro controllers)and Matlab for prototyping
   \end{itemize}}{}
    \item[$\checkmark$] Designed most of the original embedded circuitry for CAN actuator nodes
 \end{itemize}
 
 \item  \href{http://robotics.usc.edu/~ampereir/wordpress/?page_id=382}{Developed a Wave-recorder that uploaded wave-data to server via GSM cellular data-modems}
 \iftoggle{detailedVersion}{
 \begin{itemize}
 \item[$\checkmark$] One of the earliest GSM-enabled instruments in Oceanography
  \begin{itemize}
   \item[\tiny$\blacksquare$]  Code written in C (Embedded), Visual C++ for Windows-based GUI 
   \item[\tiny$\blacksquare$]  Hardware design involved PCB layout software (Eagle)
   \end{itemize}
 \end{itemize}}{}
 \end{itemize}
 
{\sl \href{http://nio.org}{\textbf{National Institute of Oceanography, Goa, India}}}       \hfill               August 2002 - August 2003 \\[2pt]
\textbf{Marine Instrumentation \& Computer Division} \hfill \textbf{Project Assistant}

  \begin{itemize}
  \item  Worked on the design of a heading controller for ROSS - "Remotely Operated Sea Skimmer", a surface craft which was remotely controlled
    \iftoggle{detailedVersion}{\begin{itemize}
  \item[$\checkmark$] Tested robot successfully at sea (even soldered broken circuitry at sea)
  \begin{itemize}
   \item[\tiny$\blacksquare$]  Code written in C/C++ (Linux), Visual C++ for Windows-based GUI
   \item[\tiny$\blacksquare$]  Hardware design involved PCB layout software (Eagle)
   \end{itemize}
 \end{itemize}}{}

   \item  Developed Tide-gauges for the Indo-Ghanaian Collaborative program (UNESCO, IOC)
  \begin{itemize}
  \item[$\checkmark$] Tide gauges developed still functional after 8 years (without servicing)
  \iftoggle{detailedVersion}{\begin{itemize}
   \item[\tiny$\blacksquare$]  Code written in C/C++ (Embedded), Visual C++ for Windows-based GUI
   \end{itemize}}{}
 \end{itemize}
 \end{itemize}

%\newpage
\vspace{0.2in}
\section{\centerline{ENTREPRENEURIAL EXPERIENCE}}
\vspace{8pt}
\begin{itemize}
\item KnoVid - Developing an alternate unobtrusive Interest-based ad system for Android and iOS devices
\item KartTime - Designed a system that determines lap-timings for Go-Karts \iftoggle{detailedVersion}{and which uses short-range wireless RF-communication based on slotted TDM, FDM and FSK}{}
\end{itemize}

\section{\centerline{MENTORING EXPERIENCE}}
\vspace{8pt}
\centerline{Undergraduate student projects mentored (student groups usually consisted of 3-5 students).}
\vspace{-8pt}
\begin{table}[ht]
\centering
\begin{tabular}{ll}
{\tiny$\blacksquare$}~PIC-based Fin/Rudder Node for AUV &  {\tiny$\blacksquare$}~PIC-based Tide Level Measurement System \\
{\tiny$\blacksquare$}~PI-controller for Thruster speed regulation & {\tiny$\blacksquare$}~Pan and Tilt system for Thrust Vectoring an AUV
\end{tabular}
\end{table}

%\iftoggle{detailedVersion}{\newpage}

\vspace{0.2in}
\section{\centerline{TECHNICAL EXPERTISE}}
\vspace{8pt}

\begin{table*}[h]
    \begin{tabular}{|lll|}
        \hline \bfseries Technology & \bfseries Proficient & \bfseries Prior Experience \\\hline 
        \hline \bfseries Programming Languages & C, C++ (OOP, \href{http://boost.org}{Boost}, STL, \href{http://beej.us/guide/bgnet/output/html/multipage/index.html}{Sockets},  & Visual C++, C\#, Java, Obj-C  \\
        & \href{https://computing.llnl.gov/tutorials/pthreads/}{Multi-threading}), \href{http://www.greenteapress.com/thinkpython/html/index.html}{Python}, \href{http://www.mathworks.com/}{Matlab}, Shell Scripting &  Javascript, SQL, Perl, \href{http://www.latex-project.org/}{\LaTeX} \\
        \hline \bfseries Operating Systems & \href{http://www.ubuntu.com/}{Ubuntu} \href{http://www.linux.org/}{Linux}, Unix, \href{https://developer.apple.com/}{OS-X}, \href{http://windows.microsoft.com/en-us/windows/home}{Windows} & \href{http://en.wikipedia.org/wiki/DOS}{MS-DOS}, \href{http://www.tinyos.net/}{TinyOS}  \\
        \hline \bfseries Frameworks & \href{https://developer.apple.com/devcenter/ios/index.action}{iOS (Xcode)}, \href{http://developer.android.com/index.html}{Android}, \href{http://www.ros.org/wiki/}{ROS}, \href{http://pointclouds.org/}{PCL}, \href{http://opencv.org/}{OpenCV} &  \href{http://playerstage.sourceforge.net/} {Player, Stage,} \href{http://gazebosim.org/}{Gazebo}, \href{http://www.openni.org/}{OpenNI}  \\
        \hline \bfseries Testing & PyUnit, \href{http://robotics.usc.edu/~ampereir/wordpress/?p=772}{CppUnit} & JUnit, NUnit, \href{https://code.google.com/p/googletest/}{googletest} \\
        \hline \bfseries Version Control & \href{http://robotics.usc.edu/~ampereir/wordpress/?p=487}{Git}, \href{http://subversion.apache.org/}{Subversion} &  \href{http://mercurial.selenic.com/}{Mercurial} \\
        \hline \bfseries Tools \& Utilities & Eclipse, Xcode, \href{http://www.mathworks.com/products/simulink/}{Simulink}, \href{http://www.numpy.org/}{NumPy}, \href{http://www.scipy.org/}{SciPy},  & SPICE simulation, \href{http://www.csie.ntu.edu.tw/~cjlin/libsvm/}{libsvm}, \href{http://www.gaussianprocess.org/gpml/code/matlab/doc/}{GPML} \\
           &  \href{http://www.microsoft.com/visualstudio/}{Visual Studio}, PCB layout and design (Eagle) & \href{https://developer.nvidia.com/category/zone/cuda-zone}{NVidia's CUDA}, \href{http://www.cadence.com/products/orcad/pages/default.aspx}{OrCAD} \\
        \hline \bfseries Databases & \href{http://www.mysql.com/}{MySQL}, SQLite, \href{http://www.oracle.com/}{Oracle} & \\
        \hline \bfseries Hardware & Intel (x86, 8051), ARM, PICs, Gumstix  & \href{http://www.clear.rice.edu/elec201/Book/6811_asm.html}{Motorola 68x}, \href{http://robotics.usc.edu/~ampereir/wordpress/?p=156}{TI (67x DSP kits)} \\
           &  TI (MSP430)  & Berkeley Motes \\
        \hline 
     \end{tabular}
   \end{table*}

 
\vspace{0.2in}
\section{\centerline{EDUCATION}}
\vspace{8pt}
{\sl PhD}, Computer Science \\
{\bf University of Southern California, CA \hspace{0.5in}  GPA 3.56 \hfill Sep 2013 (graduated)}\\
Thesis - \href{http://robotics.usc.edu/~ampereir/wordpress/?page_id=131}{Path Planning for Autonomous Underwater Vehicles in the Open Ocean}\\
Advisor - \href{http://robotics.usc.edu/~gaurav}{Prof. Gaurav S. Sukhatme}

{\sl Master of Science}, Electrical Engineering \\
{\bf University of Southern California, CA \hspace{0.5in}  GPA 3.72 \hfill May 2007 (graduated)}\\
Thesis - \href{http://robotics.usc.edu/~ampereir/wordpress/?page_id=365}{Navigation and Guidance of an Autonomous Surface Vehicle} \\
Advanced DSP project - \href{http://robotics.usc.edu/~ampereir/wordpress/?p=156}{Tracking groups of people in real-time video using TI DSP processor} \\
Sensing and Planning in Robotics Project -  \href{http://robotics.usc.edu/~ampereir/pubs/report547.pdf}{Mapless Localization from Fused Sensor Data}
 
 \vspace{0.2in}
 
{\sl Bachelor of Engineering}, Electronics and Communications \\ % \sl will be bold italic in
{\bf \href{http://www.vtu.ac.in/}{VTU}, Belgaum, Karnataka, India   \hspace{0.5in} 1st class with Distinction   \hfill    Jul 2002 (graduated)} \\
Project 1 - \href{http://robotics.usc.edu/~ampereir/wordpress/?p=382}{Development of a Automatic Weather Station} \\
Project 2 - \href{http://robotics.usc.edu/~ampereir/wordpress/?p=382}{PC-based Blood Sample Colorimeter}
 
 \vspace{0in} 

%\iftoggle{detailedVersion}{
\section{\centerline{MEMBERSHIPS}} 
\begin{itemize}
      \iftoggle{detailedVersion}{
      \item Student member of the IEEE since 2007
       \item Member of the Robotics and Automation Society since 2007 
      \item Apple iOS Developer since 2009
      \item Reviewer for Journals and Conference articles (JFR, AURO,TRO, ICRA, IROS, OCEANS)}
      {\item Student member of IEEE and Robotics and Automation Society since 2007
      \item Apple iOS Developer since 2009}
 \end{itemize}
 %}{}
 
 \vspace{0.2in}
%\iftoggle{detailedVersion}{
\section{\centerline{PUBLICATIONS}}
\iftoggle{detailedVersion}{
\bibliographystyle{arvindpereira}
\vspace{15pt}
\begingroup
\renewcommand{\section}[2]{}%
{\def\section*#1{}\bibliography{arvind_pereira}}
\endgroup
\nocite{*}}{
{\begin{itemize}
\item Recent of publications: {\url{http://robotics.usc.edu/~ampereir/publications}}
\iftoggle{detailedVersion}{\item Google Scholar Citations: {\url{http://robotics.usc.edu/~ampereir/google_scholar}}}{}
\end{itemize}
}}


 \iftoggle{detailedVersion}{
 \vspace{0.2in}
 \section{\centerline{REFERENCES}}
 \vspace{8pt}
 \begin{center}
 References provided upon request. (Online references available at \href{http://linkedin.com/in/arvindpereira}{LinkedIn}).
 \end{center}}{}

\iftoggle{detailedVersion}{
\vspace{0.5in}
\centerline{\color{blue}{(Kindly note, that the pdf version of this resume contains clickable links to more detailed information on  projects)}}
}{}
 
\end{resume} 
\end{document}

